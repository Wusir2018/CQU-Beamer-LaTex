\documentclass[hyperref,UTF8,11pt]{beamer}
\usepackage{ctex}
\usepackage[utf8]{inputenc}
\usepackage{fontspec}
\usepackage{comment}
%\setCJKfamilyfont{SimHei}
\usepackage{xeCJK}
%\renewcommand{\CJKfamilydefault}{\CJKsfdefault}
%\setmonofont{Consolas}
\setsansfont{Microsoft YaHei}
\setCJKmainfont{Microsoft YaHei}
%\setCJKmonofont{KaiTi}
%\setCJKsansfont{Microsoft YaHei}
\usefonttheme{professionalfonts}
\usepackage{hyperref}
\usepackage{graphicx}
\graphicspath{{image/}} % storage figure in a sub-folder
% \usepackage[parfill]{parskip} % Activate to begin paragraphs with an empty line rather than an indent
%\usepackage{epstopdf}
\usepackage{bm}
\usepackage{cqucolor}
%\usepackage{hyperref}
\hypersetup{CJKbookmarks=true}
\usepackage{url}
\usepackage{amsmath}
\usepackage{amsthm}
%\theoremstyle{definition}
%\newtheorem{theorem}{定理}
%\newtheorem{definition}{定义}
%\newtheorem{corollary}{推论}
%\newtheorem{example}{例}
\usepackage{booktabs} % for much better looking tables
%\usepackage{cite} % reference
\usepackage[backend=biber,style=numeric,sorting=none]{biblatex}
%\usepackage[backend=biber,style=apalike]{biblatex}
%\usepackage[backend=biber,style=authoryear]{biblatex}
% if style=apalike or authoryear, use \parencite instead of \cite
\addbibresource{cquthesis.bib}
\beamertemplatetextbibitems
\usepackage{array} % for better arrays (eg matrices) in maths
%\usepackage{paralist} % very flexible & customisable lists (eg. enumerate/itemize, etc.)
\usepackage{verbatim} % adds environment for commenting out blocks of text & for better verbatim
\usepackage{subfigure} % make it possible to include more than one captioned figure/table in a single float
% These packages are all incorporated in the memoir class to one degree or another...
%\usepackage{threeparttable}
\usepackage{cases} %equation set
\usepackage{multirow} %use table
\usepackage{enumerate}
\usepackage{algorithm}
\usepackage{algorithmic}
\usepackage{xcolor}
%\usepackage{capt-of}
\setcounter{tocdepth}{1}%只显示section,不显示subsection
\usepackage{listings}
\lstset{tabsize=4, keepspaces=true,
    xleftmargin=2em,xrightmargin=0em, aboveskip=1em,
    backgroundcolor=\color{gray!20},  % 定义背景颜色
    frame=none,                       % 表示不要边框
    extendedchars=false,              % 解决代码跨页时,章节标题,页眉等汉字不显示的问题
    numberstyle=\ttfamily,
    basicstyle=\ttfamily,
    keywordstyle=\color{blue}\bfseries,
    breakindent=10pt,
    identifierstyle=,                 % nothing happens
    commentstyle=\color{green}\small,  % 注释的设置
    morecomment=[l][\color{green}]{\#},
    numbers=left,stepnumber=1,numberstyle=\scriptsize,
    showstringspaces=false,
    showspaces=false,
    flexiblecolumns=true,
    breaklines=true, breakautoindent=true,breakindent=4em,
    escapeinside={/*@}{@*/},
}

\title[Beamer模板]{中文题目}
\subtitle{English Title}
\author[作者]{答辩人:\quad \\ 学号:\quad  \\ 专业:\quad \\ 指导教师:\quad \\\quad}
\institute[学院]{重庆大学 \quad 学院}
\date{2019年5月} %Activate to display a given date or no date (if empty),
% otherwise the current date is printed

\begin{document}
%%%%%%%%%% 定理类环境的定义 %%%%%%%%%%
%% 必须在导入中文环境之后
\newcommand{\redstress}[1]{{\color{red}{#1}}}
%\renewcommand{\raggedright}{\leftskip=0pt \rightskip=0pt plus 0cm}
\renewcommand{\contentsname}{目录}     % 将Contents改为目录
\renewcommand{\abstractname}{摘要}     % 将Abstract改为摘要
\renewcommand{\refname}{参考文献}      % 将References改为参考文献
\renewcommand{\indexname}{索引}
\renewcommand{\figurename}{图}
\renewcommand{\tablename}{表}
\renewcommand{\appendixname}{附录}
%\renewcommand{\proofname}{证明}
%\renewcommand{\algorithm}{算法}
%----------------------------------------------------------------------
% Title frame
\begin{frame}
\maketitle
\end{frame}


%%%%%%%%%%%%%%%%%%%%%%%%%%%%%%%%%%%%%%%%%%%%%%%%%%%%%%%%%%%%%%%%%

\section[Introduction ����]{Introduction}\label{sec:1}

%%%%%%%%%%%%%%%%%%%%%%%%%%%%%% ����Ŀ¼ҳ %%%%%%%%%%%%%%%%%%%%%%%%%%%%%%%%%%%

\begin{frame}%<beamer>
    \frametitle{\textsc{Contents}} \vspace{-1.05cm}
    \begin{multicols}{2}
    %\begin{figure}
    \begin{minipage}[t]{0.55\textwidth}
    \tableofcontents[currentsection,hideallsubsections]
    % [currentsection,hideallsubsections][sectionstyle=show/shaded,subsectionstyle=show/shaded/hide]
    \end{minipage}

    \begin{minipage}[t]{0.55\textwidth}
    \vspace{0.44cm}
    \begin{spacing}{1.2} % ������� ��Ҫ\usepackage{setspace}
    \begin{itemize}
    \item\hyperlink{subsec:1-1}{�������������}
    \item\hyperlink{subsec:1-2}{����취}
    %\item\hyperlink{subsec:1-3}{Symbols and Notes}
    %\item\hyperlink{subsec:1-4}{EM Spectrum}
    \end{itemize}
    \end{spacing}
    \end{minipage}
    %\end{figure}
    \end{multicols}

\end{frame}

%%%%%%%%%%%%%%%%%%%%%%%%%%%%%%%%%%%%%%%%%%%%%%%%%%%%%%%%%%%%%%%%%
\subsection[History and background]{����������}\label{subsec:1-1}
%%%%%%%%%%%%%%%%%%%%%%%%%%%%%%%%%%%%%%%%%%%%%%%%%%%%%%%%%%%%%%%%%
\begin{frame}
\frametitle{\textsc{����, ����}}%\transsplitverticalin

\begin{itemize}
\hilite<1>\item ���Բ���ϵ: ʹţ����ѧ�����IJ���ϵ\pause

\hilite<2>\item ţ����ѧ��ʱ�չ�: ٤���Ա任
\begin{align}
&t'=t,\\
&x'=x+vt,\\
&y'=y,\\
&z'=z.\pause
\end{align}
%\begin{figure}
 % \begin{center}
  %  \includegraphics[scale=0.4]{ch1/first_radio_link} % ͼƬ�������Դ�Щ��ĸ��ͷ
  %\end{center}
%\end{figure}

\hilite<3>\item ٤���������ԭ��: %ţ����ѧ����, ��һ�й���ϵ�任����.

\end{itemize}

\end{frame}

%%%%%%%%%%%%%%%%%%%%%%%%%%%%%%%%%%%%%%%%%%%%%%%%%%%%%%%%%%%%%%%%%
\begin{frame}
\frametitle{\textsc{History}} % \transsplitverticalout

\begin{itemize}
    \hilite<1>\item ����ϵ����\pause
    \hilite<2>\item �ٶ����������\pause
    \hilite<3>\item ��Ų�: ������������, ��������
\begin{align}\left.
\begin{array}{l}
\nabla\cdot\bm{E}=\frac{\rho}{\epsilon_0}\\
\nabla\times\bm{B}=\mu_0\bm{J}+\mu_0\epsilon_0\frac{\partial\bm{E}}{\partial t},\\
\nabla\cdot\bm{B}=0\\
\nabla\times\bm{E}=-\frac{\partial\bm{B}}{\partial t}\end{array}
\right\}\bm{E}=\bm{E}_0\cos(kx-\omega t),
\end{align}
\begin{align}
c=\frac{\omega}{k}=\frac{1}{\sqrt{\mu_0\epsilon_0}}\nonumber
\end{align}
   % \begin{figure}
    %\begin{center}
     %   \includegraphics[scale=0.5]{ch1/antenna_marconi} % ͼƬ�������Դ�Щ��ĸ��ͷ
    %\end{center}
    %\end{figure}
\end{itemize}
%\rightline{\hyperlink{sec:1}{\beamerreturnbutton{back}} }

\end{frame}

%%%%%%%%%%%%%%%%%%%%%%%%%%%%%%%%%%%%%%%%%%%%%%%%%%%%%%%%%%%%%%%%%
\subsection[���]{�������}\label{subsec:1-2}
%%%%%%%%%%%%%%%%%%%%%%%%%%%%%%%%%%%%%%%%%%%%%%%%%%%%%%%%%%%%%%%%%

\begin{frame}
\frametitle{\textsc{��ν��?}}%\transwipe % ͿĨЧ��

\begin{itemize}% [<+-| structure@+>]
\hilite<1>\item ���ٲ���ԭ��\pause

%\hilite<2>\item 1960s-1990s, advances made in computer architecture
%and technology have had a major impact on the advance of modern
%antenna technology, numerical methods were introduced that allowed
%previously intractable complex antenna system configurations to be
%analyzed and designed very accurately.
\hilite<2>\item ���������ԭ��: һ�����������ڲ�ͬ����ϵ����ʽ��ͬ
\end{itemize}
%\rightline{\hyperlink{sec:1}{\beamerreturnbutton{back}} }

\end{frame}

\subsection{basic frame}

\begin{frame}
\frametitle{框架:Why I made this}
%\framesubtitle{Why I made this beamer style}
\begin{block}{Demonstration of the use of items and blocks}
\begin{itemize}
\item No one has done it.$$e=mc^2$$
\item I need one.
\pause \item Share with others.
\end{itemize}
\end{block}
\pause
\begin{block}{Another block}
This block appears after a pause. Simply delete the \texttt{\textbackslash pause} command if this animation is not needed. Add the pause command whenever a pause is needed. 
\end{block}
\end{frame}

%%%%%%%%%%%%%%%%%%%%%%%%%%%%%%%%%%%%%%%%%%%%%%%%%%%%%%%%%%%%%%%%%

\section[��������۵�ʱ�չ�]{spacetime of SR}

%%%%%%%%%%%%%%%%%%%%%%%%%%%%%% ����Ŀ¼ҳ %%%%%%%%%%%%%%%%%%%%%%%%%%%%%%%%%%%

\begin{frame}%<beamer>
    \frametitle{\textsc{Contents}} \vspace{-0.85cm}\label{sec:3}
    \begin{multicols}{2}
    \begin{minipage}[t]{0.55\textwidth}
    \tableofcontents[currentsection,hideallsubsections]
    % [currentsection,hideallsubsections][sectionstyle=show/shaded,subsectionstyle=show/shaded/hide]
    \end{minipage}

    \begin{minipage}[t]{0.55\textwidth}
    \vspace{0.6cm}
    \begin{spacing}{0.9} % ������� ��Ҫ\usepackage{setspace}
    \begin{itemize}
        \item\hyperlink{subsec:3-1}{��������, ��������; ��~``��'' �ӻ�, ����������}
        \item\hyperlink{subsec:3-2}{ʱ��ͼ, �����, ͬʱ�������}
        \item\hyperlink{subsec:3-3}{�ִ����澡ͷ; ���ȥ����δ���˻���}
      %  \item\hyperlink{subsec:3-4}{Waveguide Antennas}
      %  \item\hyperlink{subsec:3-5}{Flat-Sheet Reflector Antennas}
      %  \item\hyperlink{subsec:3-6}{Radio Communication Link}
      %  \item\hyperlink{subsec:3-7}{Fields From Dipole}
      %  \item\hyperlink{subsec:3-8}{Antenna Field Zones}
      %  \item\hyperlink{subsec:3-9}{Shape-Impedance Considerations}
    \end{itemize}
    \end{spacing}
    \end{minipage}
    \end{multicols}
\end{frame}

%%%%%%%%%%%%%%%%%%%%%%%%%%%%%%%%%%%%%%%%%%%%%%%%%%%%%%%%%%%%%%%%%
\subsection[��������]{shorter and slower}\label{subsec:3-1}
%%%%%%%%%%%%%%%%%%%%%%%%%%%%%%%%%%%%%%%%%%%%%%%%%%%%%%%%%%%%%%%%%


\begin{frame}
\frametitle{\textsc{��������}}
\begin{itemize}

\hilite<1>\item �����ȱ任, ��ʹ���Ƿ���һ��ȫ�µ������; ͬʱ, ����������ͬʱ���һ��, ������һ�θ���̵ؾ�������: �й������Ǵ�dz��, ����ȷ�Ĺ��������ڸ���ȷ������Ļ��.\pause

\hilite<2>\item ���Ǽ���, $\Sigma'$ ϵ����һ����, ��Ϊ~$L'$, ��, ��~$\Sigma$ �Ͽ���, ����~$L$ �Ƕ���?
\begin{align}
x'_1=\gamma\beta ct+\gamma x_1,\\
x'_2=\gamma\beta ct+\gamma x_2,
\end{align}
��ʽ�������
\begin{align}
L'=\gamma L, or~L=\sqrt{1-\frac{v^2}{c^2}}L'.\nonumber
\end{align}

\end{itemize}

\end{frame}

%%%%%%%%%%%%%%%%%%%%%%%%%%%%%
\begin{frame}
\frametitle{\textsc{���ڳ��ȵļ�������}}
\begin{itemize}

\hilite<1>\item ���⵽���ܲ���װ�³���? ���������ͬʱ/��ͬʱ��;\pause 

\hilite<2>\item �������׻᲻����±���? �����ij��������� (�������������һֱ��, ����ͬ); ����ͬʱ/��ͬʱ��;\pause

\hilite<3>\item ��·���׻᲻�ᱻ��ͨ? �⴫����ʱ��;\pause
\hilite<4>\item DZˮͧ���ϸ������³�? ����ȫ���Ļ���������, �޸���; ������~(������׹�����): ����������, ���ɴ���ˮ���������˶�������, ���ٴ��еķɴ�, ˮ��������������, �������ɲ�����������. ǰ��˵��, �д���ȶ. �����������������, ���Կ�����˲ʱ��. ʵ�����, ˮ��ͧ�ƶ��²����α�����, ����������; ��ˮ������ˮ�������, ����˼����, �ƿ���֮ˮ/�谭��, ȴ���ܵ��κη�����, �������. ���Լ����������ˮ (��ѹ����ը), Ҳ���������������ƶ�.
\end{itemize}

\end{frame}

%%%%%%%%%%%%%%%%%%%%%%%%%%%%%%%%%%%%%%%%%%%%%%%%%%%%%%%%%%%%%%%%%


\begin{frame}
\frametitle{\textsc{��~``��'' �ӻ�, ����������}}
\begin{itemize}

\hilite<1>\item ���Ǽ�����~$\Sigma$ ϵ��ij�̶��㴦, ����һ�ν���, ������, ����
\begin{align}
ct'_1=\gamma ct_1+\beta\gamma x,\\
ct'_2=\gamma ct_2+\beta\gamma x,
\end{align}
������֪
\begin{align}
\Delta t'=\gamma \Delta t.\pause
\end{align}


\hilite<2>\item ����˭������, ˭����?\pause
\hilite<3>\item ˫���ű�ը����: �᲻��ը?

\end{itemize}

%\rightline{\hyperlink{sec:3}{\beamerreturnbutton{back}} }

\end{frame}


%%%%%%%%%%%%%%%%%%%%%%%%%%%%%%%%%%%%%%%%%%%%%%%%%%%%%%%%%%%%%%%%%








%%%%%%%%%%%%%%%%%%%%%%%%%%%%%%%%%%%%%%%%%%%%%%%%%%%%%%%%%%%%%%%%%
\subsection[ʱ��ͼ, �����, ͬʱ�������]{causality}\label{subsec:3-2}
%%%%%%%%%%%%%%%%%%%%%%%%%%%%%%%%%%%%%%%%%%%%%%%%%%%%%%%%%%%%%%%%%

\begin{frame}
\frametitle{\textsc{(ij�¼���) ʱ��ͼ, ��׶}}
\begin{figure}[!h]
\begin{center}
\includegraphics[width=4.3 cm]{figure/space.jpg}
\caption{���/ʱ������Ļ���.}
\label{space}
\end{center}
\end{figure}
\end{frame}


\begin{frame}
\frametitle{\textsc{�������¼���ij�¼��ļ���ķ���}}
\begin{itemize}
\hilite<1>\item �����~$ds^2=0$, ��׶��\pause
\hilite<2>\item ��ʱ���~$ds^2>0$, ��׶��: ����δ��, ���Թ�ȥ\pause
\hilite<3>\item ��ռ��~$dx^2<0$, ��׶��: �ص�δ��, ͻ�����?
\end{itemize}
\end{frame}


\begin{frame}
\frametitle{\textsc{�����, ��������}}
\begin{itemize}
\hilite<1>\item ����ɲ�����: �������ϵ���´���ز��ɱ�; ͬʱ���\pause
\hilite<2>\item ��ά������ά�����۲�: ����, ʱ�����\pause
\hilite<3>\item ��ά�������, ��ά����ʸ��, ��ά�ٶ�/����ʸ��, ����.
\end{itemize}
%\rightline{\hyperlink{sec:3}{\beamerreturnbutton{back}} }
\end{frame}




%%%%%%%%%%%%%%%%%%%%%%%%%%%%%%%%%%%%%%%%%%%%%%%%%%%%%%%%%%%%%%%%%
\subsection[�ִ����澡ͷ; ���ȥ����δ���˻���]{niubi}\label{subsec:3-3}
%%%%%%%%%%%%%%%%%%%%%%%%%%%%%%%%%%%%%%%%%%%%%%%%%%%%%%%%%%%%%%%%%

\begin{frame}
\frametitle{\textsc{�ִ����澡ͷ; ���ȥ����δ���˻���}}
�ִ����澡ͷ; ���ȥ����δ���˻���

\rightline{\hyperlink{sec:3}{\beamerreturnbutton{back}} }
\end{frame}

%%%%%%%%%%%%%%%%%%%%%%%%%%%%%%%%%%%%%%%%%%%%%%%%%%%%%%%%%%%%%%%%%

\section[�������ʱ�չ�����������ѧ]{�������ʱ�չ�����������ѧ}

%%%%%%%%%%%%%%%%%%%%%%%%%%%%%% ����Ŀ¼ҳ %%%%%%%%%%%%%%%%%%%%%%%%%%%%%%%%%%%

\begin{frame}%<beamer>
    \frametitle{\textsc{Contents}} \vspace{-1.05cm}\label{sec:4}
    \begin{multicols}{2}
    \begin{minipage}[t]{0.55\textwidth}
    \tableofcontents[currentsection,hideallsubsections]
    % [currentsection,hideallsubsections][sectionstyle=show/shaded,subsectionstyle=show/shaded/hide]
    \end{minipage}

    \begin{minipage}[t]{0.55\textwidth}
    \vspace{1.0cm}
    \begin{spacing}{1.05} % ������� ��Ҫ\usepackage{setspace}
    \begin{itemize}
        \item\hyperlink{subsec:4-1}{�������ѧ: ����άЭ���������ܵȼ�}
        \item\hyperlink{subsec:4-2}{���ѧ���������Э��}
      %  \item\hyperlink{subsec:4-3}{Power Theorem}
       % \item\hyperlink{subsec:4-4}{Radiation Intensity}
       % \item\hyperlink{subsec:4-5}{Examples of Power Patterns}
       % \item\hyperlink{subsec:4-6}{Field Patterns}
       % \item\hyperlink{subsec:4-7}{Phase Patterns}
    \end{itemize}
    \end{spacing}
    \end{minipage}
    \end{multicols}
\end{frame}

%%%%%%%%%%%%%%%%%%%%%%%%%%%%%%%%%%%%%%%%%%%%%%%%%%%%%%%%%%%%%%%%
\subsection[�������ѧ: ����άЭ���������ܵȼ�]{mass-energy equivalence}\label{subsec:4-1}
%%%%%%%%%%%%%%%%%%%%%%%%%%%%%%%%%%%%%%%%%%%%%%%%%%%%%%%%%%%%%%%%

\begin{frame}
\frametitle{\textsc{��ά�ٶ�ʸ��}}
\begin{itemize}
\hilite<1>\item ����������άʸ����, ��򵥵����ٶ���άʸ�� ���߶���Ϊ
\begin{align}
V^\mu:=\frac{dx^\mu}{d\tau}=\gamma\frac{dx^\mu}{dt}=\gamma(c,\bm{V}).\pause
\end{align}
\hilite<2>\item Ϊʲô������? ��֤: ����ʸ���ϳ�һ������:
\begin{align}
V^\mu V_\mu=\gamma^2c^2-\gamma^2 V^2=c^2.\nonumber
\end{align}

\end{itemize}
%\rightline{\hyperlink{sec:4}{\beamerreturnbutton{back}} }
\end{frame}






\begin{frame}
\frametitle{\textsc{��ά����ʸ��}}
\begin{itemize}

\hilite<1>\item ����, ������άʸ�����Ƕ�������
\begin{align}
p^\mu:=&m_0V^\mu=(\gamma m_0c,\gamma m_0\bm{V})\nonumber\\
=&(\frac{\gamma m_0c^2}{c},\bm{p})=(\frac{E}{c},\bm{p});\pause
\end{align}
\hilite<2>\item ��ʽ��ȡ $E=\gamma m_0c^2$ ��ԭ��, ����̩��չ�����Է���~$\gamma m_0c^2=m_0c^2+\frac{1}{2}m_0v^2+\cdots$.




\end{itemize}
%\rightline{\hyperlink{sec:4}{\beamerreturnbutton{back}} }
\end{frame}









\begin{frame}
\frametitle{\textsc{ΰ������ܵȼ�����}}
\begin{itemize}

\hilite<1>\item �������Ƿ���, �����徲ֹʱ, ����
\begin{align}
E_0=m_0c^2
\end{align}
������, ���Ϊ���ܵȼ�;\pause

\hilite<2>\item ��~$\frac{E^2}{c^2}-\bm{p}^2=m_0^2c^2$, ��
\begin{align}
E^2=\bm{p}^2c^2+m_0^2c^4,\nonumber
\end{align}
��Ϊ�ܶ���ϵ.

\end{itemize}
%\rightline{\hyperlink{sec:4}{\beamerreturnbutton{back}} }
\end{frame}






%%%%%%%%%%%%%%%%%%%%%%%%%%%%%%%%%%%%%%%%%%%%%%%%%%%%%%%%%%%%%%%%
\subsection[���ѧ���������Э��]{���ѧ���������Э��}\label{subsec:4-2}
%%%%%%%%%%%%%%%%%%%%%%%%%%%%%%%%%%%%%%%%%%%%%%%%%%%%%%%%%%%%%%%%

\begin{frame}
\frametitle{\textsc{���˹Τ���̵�������Э����ʽ}}
\begin{itemize}
\hilite<1>\item ��: ���ѧ����, �����˹Τ����, �Ƿ�������Э��?\pause
\hilite<2>\item ���˹Τ���̿�дΪ
\begin{gather}
\partial_\mu F^{\mu\nu}=\mu_0J^\nu,\\
\partial_\mu F_{\nu\rho}+\partial_\nu F_{\rho\mu}+\partial_\rho F_{\mu\nu}=0;
\end{gather}
�����ѧ������������άЭ���.

\end{itemize}
%\rightline{\hyperlink{sec:4}{\beamerreturnbutton{back}} }
\end{frame}

\begin{frame}
\frametitle{\textsc{���˹Τ���̵�������Э����ʽ}}
\begin{itemize}
\hilite<1>\item �����������?\pause
\hilite<2>\item Welcome to the lessons of ͯУ��!

\end{itemize}
%\rightline{\hyperlink{sec:4}{\beamerreturnbutton{back}} }
\end{frame}






\begin{frame}{致谢}
    \begin{block}{致谢}
        谢谢大家。
    \end{block}
\end{frame}

\begin{frame}[allowframebreaks]
    \frametitle{参考文献}
    \printbibliography
\end{frame}


\end{document}
