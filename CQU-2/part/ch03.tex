%%=====================================================================
% Section II
\section{extend usage}
%----------------------------------------------------------------------
\subsection{format}
\begin{frame}
\frametitle{A Two-column Slide}
\begin{columns} 
\begin{column}{0.5\textwidth} 
\begin{block}{The first column}
\begin{figure}[htb]
	\includegraphics[width=2.5cm,height=1.3cm]{figure/nkpurple.png}
	\caption{插入图片示例}
	\label{fig1}
    \end{figure}
\end{block}
\end{column}
\begin{column}{0.5\textwidth} 
\begin{block}{The second column}
颜色如图\ref{fig1},以及 e.g. {\color{red}{red}}, {\color{orange}{orange}}, {\color{blue}{blue}}
\vspace{9.5em}
\end{block}
\end{column}
\end{columns}
\end{frame}


\begin{frame}
    \frametitle{无序列表}
    \begin{description}
        \item[i] first of all
        \item[ii] besides
        \item[iii] last but not least
    \end{description}
    \begin{equation}
        \text{e}^{\pi \text{j}} + 1 = 0
    \end{equation}
    \begin{itemize}
        \item first
        \item second
    \end{itemize}
\end{frame}

\begin{frame}{表格}
\begin{table}[!hbp]
\centering
\begin{tabular}{c|c}
	\hline
	甲 &乙\\
	\hline
	11 & 12\\
	21 & 22\\
	31 & 32\\
	\hline
\end{tabular}
\caption{插入表格示例}
\label{tab1}
\end{table}
\end{frame}

\begin{frame}[fragile]
    \frametitle{code highlight}
    \begin{lstlisting}[language=java]
    public class hello{
        public static void main(String args[]){
            System.out.println("hello,world");
        }
    }
    \end{lstlisting}
\end{frame}

\begin{frame}
    \frametitle{theorem and proof}
    \begin{theorem}[L\'{e}vy\index{L\'{e}vy 定理}]
    令 $F(x),\varphi(t)$ 分别为随机变量 $X$ 的分布函数和特征函数。
    假定 $F(x)$ 在 $a+h$ 和 $a-h (h>0)$ 处连续,则有
    \begin{align}
    \label{Levy theorem}  % 方程的标记可以是专有名词
    F(a+h)-F(a-h)&=\lim_{T\rightarrow\infty}\frac{1}{\pi}\int^{T}_{-T}\frac{\sin ht}{t}
    e^{-ita}\varphi(t)dt
    \end{align}
    \end{theorem}
    \begin{proof}
        略。
    \end{proof}
\end{frame}